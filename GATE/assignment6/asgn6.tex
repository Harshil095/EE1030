\let\negmedspace\undefined
\let\negthickspace\undefined
\documentclass[journal]{IEEEtran}
\usepackage[a5paper, margin=10mm, onecolumn]{geometry}
%\usepackage{lmodern} % Ensure lmodern is loaded for pdflatex
\usepackage{tfrupee} % Include tfrupee package

\setlength{\headheight}{1cm} % Set the height of the header box
\setlength{\headsep}{0mm}     % Set the distance between the header box and the top of the text

\usepackage{gvv-book}
\usepackage{gvv}
\usepackage{cite}
\usepackage{amsmath,amssymb,amsfonts,amsthm}
\usepackage{algorithmic}
\usepackage{graphicx}
\usepackage{textcomp}
\usepackage{xcolor}
\usepackage{txfonts}
\usepackage{listings}
\usepackage{enumitem}
\usepackage{mathtools}
\usepackage{gensymb}
\usepackage{comment}
\usepackage[breaklinks=true]{hyperref}
\usepackage{tkz-euclide} 
\usepackage{listings}
% \usepackage{gvv}                                        
\def\inputGnumericTable{}                                 
\usepackage[latin1]{inputenc}                                
\usepackage{color}                                            
\usepackage{array}                                            
\usepackage{longtable}                                       
\usepackage{calc}                                             
\usepackage{multirow}                                         
\usepackage{hhline}                                           
\usepackage{ifthen}                                           
\usepackage{lscape}
\begin{document}
\bibliographystyle{IEEEtran}
\vspace{3cm}

\title{MA - 2015}
\author{EE24BTECH11064 - Harshil Rathan}
\maketitle

\renewcommand{\thefigure}{\theenumi}
\renewcommand{\thetable}{\theenumi}

\begin{enumerate}
\item Let $\tau_1$ be the usual topology on $\mathbb{R}$. Let $\tau_2$ be the topology on $\mathbb{R}$ generated by \[
\mathcal{B} = \{ (a, b) \subset \mathbb{R} : -\infty < a < b < \infty \}.
\]
Then the set 
\[
\{ x \in \mathbb{R} : 4 \sin^2 x \leq 1 \} \cup \left\{ \frac{n\pi}{2} \right\}_{n \in \mathbb{Z}}
\]
is
\begin{enumerate}
    \item closed in $(\mathbb{R}, \tau_1)$ but NOT in $(\mathbb{R}, \tau_2)$
    \item closed in $(\mathbb{R}, \tau_2)$ but NOT in $(\mathbb{R}, \tau_1)$
    \item closed in both $(\mathbb{R}, \tau_1)$ and $(\mathbb{R}, \tau_2)$
    \item neither closed in $(\mathbb{R}, \tau_1)$ nor closed in $(\mathbb{R}, \tau_2)$
\end{enumerate}
\vspace{0.5cm}
\item Let $X$ be a connected topological space such that there exists a non-constant continuous function 
$f : X \to \mathbb{R}$,
where $\mathbb{R}$ is equipped with the usual topology. Let 
$f(X) = \{ f(x) : x \in X \}$.
Then
\begin{enumerate}
      \item $X$ is countable but $f\brak{X}$ is uncountable
        \item $f\brak{X}$ is countable but $X$ is uncountable
        \item both $f\brak{X}$ and $X$ are countable
        \item both $f\brak{X}$ and $X$ are uncountable
\end{enumerate}
\vspace{0.5cm}
\item Let $d_1$ and $d_2$ denote the usual metric and the discrete metric on $\mathbb{R}$, respectively. Let $f: \brak{\mathbb{R},d_1}\rightarrow \brak{\mathbb{R},d_2}$ be defined by $f\brak{x}=x,x\in \mathbb{R}$. Then 
\begin{enumerate}
    \item $f$ is continuous but $f^{-1}$ is NOT continuous
    \item $f^{-1}$ is continuous but $f$ is NOT continuous
    \item both $f$ and $f^{-1}$ are continuous 
    \item neither $f$ nor $f^{-1}$ is continuous
\end{enumerate}
\vspace{0.5cm}
\item If the trapezoidal rule iwth single interval $[0,1]$ is exact for approximating the integral $\int_0^{1}\brak{x^3-cx^2}dx$, then the value of c is equal to \underline{\hspace{2cm}}.
\vspace{0.5cm}
\item Suppose that the Newton-Raphson method is applied to the equation $2x^2+1-e^{x^2}=0$ with an initial approximation $x_0$ sufficiently close to zero. Then, for the root x=0, the order of convergence of the method is equal to \underline{\hspace{2cm}}.

\vspace{0.5cm}
\item The maximum possible order of a homogeneous linear ordinary differential equation with real constant coefficients having $x^2\sin{x}$ as a solution is equal to \underline{\hspace{2cm}}.

\vspace{0.5cm}
\item The Lagrangian of a system in terms of polar coordinates $\brak{r,\theta}$ is given by
\begin{align*}
   L=\frac{1}{2}mr^2+\frac{1}{2}m\brak{r^2+r^2\theta^2}-mgr\brak{1-\cos{\theta}}
\end{align*}
where m is the mass, g is the acceleration dur to gravity and $s'$ denotes the derivative of $s$ with respect to time. Then the equation of motion are
\begin{enumerate}
    \item $2r'=r\theta'^2-g\brak{1-\cos{\theta}}$, $\frac{d}{dt}\brak{r^2\theta'}=-gr\sin{\theta}$
    \item  $2r'=r\theta^2+g\brak{1-\cos{\theta}}$, $\frac{d}{dt}\brak{r^2\theta}=-gr\sin{\theta}$
    \item  $2r'=r\theta^2-g\brak{1-\cos{\theta}}$, $\frac{d}{dt}\brak{r^2\theta}=gr\sin{\theta}$
    \item  $2r'=r\theta^2+g\brak{1-\cos{\theta}}$, $\frac{d}{dt}\brak{r^2\theta}=gr\sin{\theta}$
\end{enumerate}
\vspace{0.5cm}
\item If $y\brak{x}$ satisfies the initial value problem 
\begin{align*}
    \brak{x^2+y}dx-xdy, y\brak{1}=2
\end{align*}
then $y\brak{2}$ is equal to \underline{\hspace{2cm}}.
\vspace{0.5cm}

\item It is known that Bessel functions $J_n\brak{x}$, for $n\geq 0$, satisfy the identity 
\begin{align*}
    e^{\frac{x}{2}\brak{t-\frac{1}{t}}}=J_0\brak{x}+\sum_{n=1}^\infty J_n\brak{x}\brak{t^n+\frac{\brak{-1}^n}{t^n}}
\end{align*}
for all $t>0$ and $x \in \mathbb{R}$. The value of $J_0\brak{\frac{\pi}{3}}+2\sum_{n=1}^\infty J_{2n}\brak{\frac{\pi}{3}}$ is equal to
\vspace{0.5cm}

\item Let $X$ and $Y$ be two random variables having the joint probability density function
\begin{align*}
    f(x, y) = \begin{cases}
    2 & \text{if } 0 < x < y < 1 \\
    0 & \text{otherwise}
\end{cases}
\end{align*}
Then the conditional probability $P\brak{X\leq \frac{2}{3}| Y-\frac{3}{4}}$ is equal to
\begin{multicols}{2}
\begin{enumerate}
    \item[(A)] $\frac{5}{9}$
    \item[(B)] $\frac{2}{3}$
    \item[(C)] $\frac{7}{9}$
    \item[(D)] $\frac{8}{9}$
\end{enumerate}
\end{multicols}

\vspace{0.5cm}

\item Let $\ohm =(0,1]$ be the sample space and let $P\brak{}$ be a probability fucntion defined by 
\begin{align*}
    P((0, x]) = \begin{cases}
    \frac{x}{2} & \text{if } 0 \leq x < \frac{1}{2} \\
    x & \text{if} \frac{1}{2}\leq x \leq 1
\end{cases}
\end{align*}
Then $P\brak{{{\frac{1}{2}}}}$ is equal to \underline{\hspace{2cm}}.
\vspace{0.5cm}

\item Let $X_1,X_2$ and $X_3$ be independent and identically distributed random variables with $E(X_1)=0$ and $E(X_1^2=\frac{15}{4})$. If $\phi:\brak{0,\infty \rightarrow \brak{0, \infty}}$ si deifned through the conditional expectation 
\begin{align*}
    \phi{t}=E(X_1^2|X_1^2+X_2^2+X_3^2 = t) , t>0
\end{align*}
then $E(\phi((X_1+X_2)^2))$ is equal to \underline{\hspace{2cm}}.

\vspace{0.5cm}

\item Let $X \sim$ Poisson($\lambda$), where $\lambda>0$ is unknown. If $\delta\brak{X}$ is the unbiased estimator of $g\brak{\lambda}=e^{-\lambda}\brak{3\lambda^2+2\lambda+1}$, then $\sum_{k=0}^\infty\delta\brak{k}$ is equal to \underline{\hspace{2cm}}.
\end{enumerate}
\end{document}

