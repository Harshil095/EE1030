\let\negmedspace\undefined
\let\negthickspace\undefined
\documentclass[journal]{IEEEtran}
\usepackage[a5paper, margin=10mm, onecolumn]{geometry}
%\usepackage{lmodern} % Ensure lmodern is loaded for pdflatex
\usepackage{tfrupee} % Include tfrupee package

\setlength{\headheight}{1cm} % Set the height of the header box
\setlength{\headsep}{0mm}     % Set the distance between the header box and the top of the text

\usepackage{gvv-book}
\usepackage{gvv}
\usepackage{cite}
\usepackage{amsmath,amssymb,amsfonts,amsthm}
\usepackage{algorithmic}
\usepackage{graphicx}
\usepackage{textcomp}
\usepackage{xcolor}
\usepackage{txfonts}
\usepackage{listings}
\usepackage{enumitem}
\usepackage{mathtools}
\usepackage{gensymb}
\usepackage{comment}
\usepackage[breaklinks=true]{hyperref}
\usepackage{tkz-euclide} 
\usepackage{listings}
% \usepackage{gvv}                                        
\def\inputGnumericTable{}                                 
\usepackage[latin1]{inputenc}                                
\usepackage{color}                                            
\usepackage{array}                                            
\usepackage{longtable}                                       
\usepackage{calc}                                             
\usepackage{multirow}                                         
\usepackage{hhline}                                           
\usepackage{ifthen}                                           
\usepackage{lscape}
\begin{document}

\bibliographystyle{IEEEtran}
\vspace{3cm}

\title{9-9.2-10}
\author{EE24BTECH11064 - Harshil Rathan}
% \maketitle
% \newpage
% \bigskip
{\let\newpage\relax\maketitle}

\renewcommand{\thefigure}{\theenumi}
\renewcommand{\thetable}{\theenumi}
\setlength{\intextsep}{10pt} % Space between text and floats


\numberwithin{equation}{enumi}
\numberwithin{figure}{enumi}
\renewcommand{\thetable}{\theenumi}
\textbf{Question}:\\
Find the area of the region bounded by the curve $y^2= x$ and the lines $x = 1$ and
$x = 4$ and the axis in the first quadrant.\\
\solution \\
\begin{table}[h!]
    \centering
    \begin{center}
    \begin{tabular}{|c|c|c|} 
        \hline
            \textbf{Variable} & \textbf{Parameter} & \textbf{Value} \\ 
        \hline
            $\boldsymbol{BC}$ & a & 7 cm \\ 
        \hline
            $\boldsymbol{AB}$ & c & - \\ 
        \hline
            $\boldsymbol{AC}$ & b &   -    \\
        \hline
            $\angle \vec{B}$  & -  & $45^\circ$ \\
        \hline
	    $\angle \vec{C}$ & - & $60^\circ$ \\
	\hline 
    \end{tabular}
\end{center}  



\end{table}
The parameters of the conic are\\
\begin{table}[h!]
    \centering
    \begin{tabular}[12pt]{ |c| c|}
    \hline
    \textbf{Conic}& \textbf{Parameters}\\ 
    \hline
     $V$& $\myvec{1 & 0 \\ 0 & 1}$\\
    \hline 
     $u$& $0$\\
    \hline
     $f$& $-4$\\
     \hline
     $h$& $0$\\
     \hline
     $m$& $\myvec{1\\\frac{1}{\sqrt{3}}}$\\
     \hline
    \end{tabular}

    \label{Table2}
\end{table}
\begin{align}
    L : x_i=h+\kappa_i m 
\end{align}
Where,
\begin{align}
    \kappa_i=\frac{1}{m^\top Vm}(-m^\top(Vh+u) \pm \sqrt{[m^\top(Vh+u)]^2-g(h)(m^\top Vm)}
    \label{0.2}
\end{align}
For the line $x-1=0$, the parameters are\\
\begin{align}    
    h_2=\myvec{1\\0} , m_2=\myvec{0\\1}
\end{align}
Substituting from the above in (\ref{0.2})\\
\begin{align}
    \kappa_i= 1,-1
\end{align}
yilelding the points of intersection\\
\begin{align}
  a_o=\myvec{1\\1},a_1=\myvec{1\\-1}
\end{align}
Similarly, for the line $x-4=0$ (\ref{0.2})\\
\begin{align}
   h_1=\myvec{4\\0} , m_1=\myvec{0\\1}
\end{align}
yielding\\
\begin{align}
    \kappa_i= 2,-2
\end{align}
from which, the points of intersection are\\
\begin{align}
      a_3=\myvec{4\\2},a_2=\myvec{4\\-2}
\end{align}
Thus, the area of the parabola in between the lines $x=1$ and $x=4$ is given by \\
\begin{align}
    \int_{0}^{4} \sqrt{x} \, dx - \int_{0}^{1} \sqrt{x} \, dx = \frac{14}{3}
\end{align}
Area enclosed is $\frac{14}{3}$.
\begin{figure}[h!]
   \centering
   \includegraphics[width=\linewidth]{figs/Figure_1.png}
   \caption{}
   \label{stemplot}
\end{figure}





\end{document}
