%ffalse
\let\negmedspace\undefined
\let\negthickspace\undefined
\documentclass[journal,12pt,twocolumn]{IEEEtran}
\usepackage{cite}
\usepackage{amsmath,amssymb,amsfonts,amsthm}
\usepackage{algorithmic}
\usepackage{graphicx}
\usepackage{textcomp}
\usepackage{xcolor}
\usepackage{txfonts}
\usepackage{listings}
\usepackage{enumitem}
\usepackage{mathtools}
\usepackage{gensymb}
\usepackage{comment}
\usepackage[breaklinks=true]{hyperref}
\usepackage{tkz-euclide} 
\usepackage{listings}
\usepackage{gvv}                                        
%\def\inputGnumericTable{}                                 
\usepackage[latin1]{inputenc}                                
\usepackage{color}                                            
\usepackage{array}                                            
\usepackage{longtable}                                       
\usepackage{calc}                                             
\usepackage{multirow}                                         
\usepackage{hhline}                                           
\usepackage{ifthen}                                           
\usepackage{lscape}
\usepackage{tabularx}
\usepackage{array}
\usepackage{float}


\newtheorem{theorem}{Theorem}[section]
\newtheorem{problem}{Problem}
\newtheorem{proposition}{Proposition}[section]
\newtheorem{lemma}{Lemma}[section]
\newtheorem{corollary}[theorem]{Corollary}
\newtheorem{example}{Example}[section]
\newtheorem{definition}[problem]{Definition}
\newcommand{\BEQA}{\begin{eqnarray}}
\newcommand{\EEQA}{\end{eqnarray}}
\newcommand{\define}{\stackrel{\triangle}{=}}
\theoremstyle{remark}
\newtheorem{rem}{Remark}

% Marks the beginning of the document
\begin{document}
\bibliographystyle{IEEEtran}
\vspace{3cm}

\title{Assignment-1}
\author{ee24btech11064 - Harshil Rathan}
\maketitle
\newpage
\bigskip

\renewcommand{\thefigure}{\theenumi}
\renewcommand{\thetable}{\theenumi}
\begin{enumerate}
\item[4.] Let f be a function defined on $R$ (the set of all real numbers) such that $f'(x)$=2010 (x-2009) (x-2010)$^2$ (x-2011)$^3$ (x-2012)$^4$ for all x $\in R$ \\ If g is a function defined on $R$ with values in the interval (0, $\infty$) such that \\

\hspace{1cm}  $f(x)$=$ln (g(x)$, for all x $\in R$\\ then the number of points in R at which $g$ has a local maximum is \hfill (2010)\\
\item[5.] let $f$:$IR \rightarrow IR$ be defined as $f(x)$= $|x|$ + $|x^2-1|$. The total number of points at which $f$attains either a local maximum or a local minimum is\\ \hfill (2012) \\
\item[6.]Let $p(x)$ be a real polynomial of least degree which has a local maximum at x=1 and local minimum at x=3.If $p(1)$=6 and $p(3)$=2, then $p'(0)$ is \hfill (2012) \\
\item[7.]A vertical line passing through the point(h,0) intersects the ellipse $\frac{x^2}{4}$ + $\frac{y^2}{3}$ = 1 at the points P and Q .Let the tangents to the ellipse at P and Q meet at the point R. If $\Delta$(h)=area of the triangle PQR, $\Delta_1  = \max _{1/2\leq h \leq 1}\Delta\brak{h}$ and $\Delta_2 = min_{1/2\leq h \leq 1}\Delta\brak{h}$,then $\frac{8}{5}\Delta_1 - 8\Delta_2$=\\\hfill $(JEEAdv.2013)$\\
\item[8.]The slope of the tangent to the curve ($y-x^5)^2$ = x($1+x^2)^2$ at the point (1,3) is \hfill ($JEE Adv. 2014)$ \\
\item[9.]A cylindrical container is to be made from a certain solid material with the following constraints:It has a fixed inner volume of V $mm^3$, has a 2mm thick solid wall and is open at the top.The bottom of the container is a solid circular disc of thickness 2mm and is of radius equal to the outer radius of the container.\\If the volume of the material used to make the container is minimum when the inner radius of the container is 10mm, then the value of $\frac{V}{250\pi}$ is \hfill ($JEE Adv. 2015)$ \\  
\end{enumerate}
\end{document}
