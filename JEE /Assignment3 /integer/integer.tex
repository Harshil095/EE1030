%iffalse
\let\negmedspace\undefined
\let\negthickspace\undefined
\documentclass[journal,12pt,twocolumn]{IEEEtran}
\usepackage{cite}
\usepackage{amsmath,amssymb,amsfonts,amsthm}
\usepackage{algorithmic}
\usepackage{graphicx}
\usepackage{textcomp}
\usepackage{xcolor}
\usepackage{txfonts}
\usepackage{listings}
\usepackage{enumitem}
\usepackage{mathtools}
\usepackage{gensymb}
\usepackage{comment}
\usepackage[breaklinks=true]{hyperref}
\usepackage{tkz-euclide}
\usepackage{listings}
\usepackage{gvv}                                        
%\def\inputGnumericTable{}                                
\usepackage[latin1]{inputenc}                                
\usepackage{color}                                            
\usepackage{array}                                            
\usepackage{longtable}                                      
\usepackage{calc}                                            
\usepackage{multirow}                                        
\usepackage{hhline}                                          
\usepackage{ifthen}                                          
\usepackage{lscape}
\usepackage{tabularx}
\usepackage{array}
\usepackage{float}
\usepackage{multicol}


\newtheorem{theorem}{Theorem}[section]
\newtheorem{problem}{Problem}
\newtheorem{proposition}{Proposition}[section]
\newtheorem{lemma}{Lemma}[section]
\newtheorem{corollary}[theorem]{Corollary}
\newtheorem{example}{Example}[section]
\newtheorem{definition}[problem]{Definition}
\newcommand{\BEQA}{\begin{eqnarray}}
\newcommand{\EEQA}{\end{eqnarray}}
\newcommand{\define}{\stackrel{\triangle}{=}}
\theoremstyle{remark}
\newtheorem{rem}{Remark}

% Marks the beginning of the document
\begin{document}
\bibliographystyle{IEEEtran}
\vspace{3cm}

\title{2021-February Session-02-24-2021-shift-1}
\author{ee24btech11064 - Harshil Rathan}
\maketitle
\newpage
\bigskip

\renewcommand{\thefigure}{\theenumi}
`\renewcommand{\thetable}{\theenumi}
\begin{enumerate}
\item Let 
\begin{align*}
    P=\myvec{3 & -1 & -2 \\ 2 & 0 & \alpha \\ 3 &-5 &0}
\end{align*}
where $\alpha \in R$. Suppose $Q=[q_{ij}]$ is a matrix satisfying $PQ=kI_3$ for some non-zero $k\in R$. If $q_{23}=\frac{-k}{8}$ and $|Q|=\frac{k^2}{2}$, then $\alpha^2+k^2$ is equal to 
\bigskip
\item Let $B_i \brak{i=1,2,3}$ be three independent events in a sample space. The probability that only $B_1$ occur is $\alpha$, only $B_2$ occurs is $\beta$ and only $B_3$ occurs is $\gamma$. Let p be the probability that none of the events $B_i$ occurs and these 4 probabilities satisfy the equations $\brak{\alpha-2\beta}p=\alpha\beta$ and $\brak{\beta-3\gamma}p=2\beta\gamma$. All the probabilities are assumed to lie in the interval $\brak{0,1}$. Then $\brak{\frac{P\brak{B_1}}{P\brak{B_3}}}$ is equal to 
\bigskip
\item The minimum value of $\alpha$ for which the equation $\frac{4}{\sin{x}}+\frac{1}{1-\sin{x}}=\alpha$ has at least one solution in $\brak{0,\frac{\pi}{2}}$ is 
\bigskip
\item If one of the diameters of the circle $x^2+y^2-2x-6y+6=0$ is a chord of another circle 'C' whose centre is at $\brak{2,1}$, then its radius is 
\bigskip
\item $\lim_{n \to \infty} \tan \left( \sum_{r=1}^{n} \tan^{-1} \left( \frac{1}{1 + r + r^2} \right) \right)$ is equal to 
\bigskip
\item If 
\begin{align*}
    \int_{-a}^{a} \brak{|x| + |x - 2|}dx = 22
\end{align*}
and $[x]$ denotes the greatest integer $\leq x$, then $\int_{a}^{-a} \brak{x+[x]}dx$ is equal to 
\bigskip
\item Let three vectors $\vec{a},\vec{b}$ and $\vec{c}$ be such that $\vec{c}$ is coplanar with $\vec{a}$ and $\vec{b}$, $\vec{a}\cdot\vec{c}=7$ and $\vec{b}$ is perpendicular to $\vec{c}$, where $\vec{a}=-\hat{i}+\hat{j}+\hat{k}$ and $\vec{b}=2\hat{i}+\hat{k}$, then the value of $2|\vec{a}+\vec{b}+\vec{c}|^2$ is 
\bigskip
\item Let $A=\{n\in N: \text{n is a 3-digit number}\}$, $B=\{9k+2: k\in N\}$ and $C=\{9k+I:k\in N\}$ for some I $\brak{0<I<9}$. If the sum of all the elements of the set $A\cap\brak{B\cup C}$ is $274\times400$, then $I$ is equal to 
\bigskip
\item If the least and the largest real values of $\alpha$, for which the equation $z+\alpha|z-1|+2i=0$ $\brak{z \in\text{C and i}=\sqrt{-1}}$ has a solution, are p and q respectively; then $4\brak{p^2+q^2}$ is equal to 
\bigskip
\item Let $M$ be any $3\times3 $ matrix with entries from the set $\{0,1,2\}$. The maximum number of such matrices, for which the sum of diagonal elements of $M^TM$ is seven, is 
\bigskip
\end{enumerate}
\end{document}
