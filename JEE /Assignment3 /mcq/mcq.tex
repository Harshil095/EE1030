%iffalse
\let\negmedspace\undefined
\let\negthickspace\undefined
\documentclass[journal,12pt,twocolumn]{IEEEtran}
\usepackage{cite}
\usepackage{amsmath,amssymb,amsfonts,amsthm}
\usepackage{algorithmic}
\usepackage{graphicx}
\usepackage{textcomp}
\usepackage{xcolor}
\usepackage{txfonts}
\usepackage{listings}
\usepackage{enumitem}
\usepackage{mathtools}
\usepackage{gensymb}
\usepackage{comment}
\usepackage[breaklinks=true]{hyperref}
\usepackage{tkz-euclide}
\usepackage{listings}
\usepackage{gvv}                                        
%\def\inputGnumericTable{}                                
\usepackage[latin1]{inputenc}                                
\usepackage{color}                                            
\usepackage{array}                                            
\usepackage{longtable}                                      
\usepackage{calc}                                            
\usepackage{multirow}                                        
\usepackage{hhline}                                          
\usepackage{ifthen}                                          
\usepackage{lscape}
\usepackage{tabularx}
\usepackage{array}
\usepackage{float}
\usepackage{multicol}


\newtheorem{theorem}{Theorem}[section]
\newtheorem{problem}{Problem}
\newtheorem{proposition}{Proposition}[section]
\newtheorem{lemma}{Lemma}[section]
\newtheorem{corollary}[theorem]{Corollary}
\newtheorem{example}{Example}[section]
\newtheorem{definition}[problem]{Definition}
\newcommand{\BEQA}{\begin{eqnarray}}
\newcommand{\EEQA}{\end{eqnarray}}
\newcommand{\define}{\stackrel{\triangle}{=}}
\theoremstyle{remark}
\newtheorem{rem}{Remark}

% Marks the beginning of the document
\begin{document}
\bibliographystyle{IEEEtran}
\vspace{3cm}

\title{2021-February Session-02-24-2021-shift-1}
\author{ee24btech11064 - Harshil Rathan}
\maketitle
\newpage
\bigskip

\renewcommand{\thefigure}{\theenumi}
`\renewcommand{\thetable}{\theenumi}
\begin{enumerate}
\item The value of $-^{15}C_{1}+2^{15}C_{2}-3^{15}C_{3}+\cdots -15^{15}C_{1}+^{14}C_{1}+^{14}C_{3}+^{14}C_{5}+\cdots^{14}C_{11}$ is 
\begin{multicols}{2}
\begin{enumerate}
    \item $2^{14}$
    \item $2^{13}-13$
    \item $2^{16}-1$
    \item $2^{13}-14$ 
\end{enumerate}
\end{multicols}
\bigskip
\item An ordinary dice is rolled for a certain number of times. If the probability of getting an odd number $2$ times is equal to the probability of getting an even number $3$ times, then the probability of getting an odd number for odd number of times is: 
\begin{multicols}{2}
\begin{enumerate}
    \item $\frac{3}{16}$
    \item $\frac{1}{2}$
    \item $\frac{5}{16}$
    \item $\frac{1}{32}$
\end{enumerate}
\end{multicols}
\bigskip
\item Let p and q be two positive number such that $p+q=2$ and $p^4+q^4=272$. Then p and q are roots of the equation:
\begin{multicols}{2}
\begin{enumerate}
       \item $x^2-2x+2$
       \item $x^2-2x+8$
       \item $x^2-2x+136$
       \item $x^2-2x+16$
\end{enumerate}
\end{multicols}
\bigskip
\item If 
\begin{align*}
    e^{\left(\cos^2{x} + \cos^4{x} + \cos^6{x} + \cdots \right) \log_e 2}
\end{align*} 
satisfies the equation $t^2-9t+8=0$, then the value of $\frac{2\sin{x}}{\sin{x}+\sqrt{3}\cos{x}}$ $0<x<\frac{\pi}{2}$ is :
\begin{multicols}{2}
\begin{enumerate}
    \item $\frac{3}{2}$
    \item $2\sqrt{3}$
    \item $\frac{1}{2}$
    \item $\sqrt{3}$
\end{enumerate}
\end{multicols}
\bigskip
\item If the system of linear equations
\begin{align*}
    3x-2y-kz=10
\end{align*}
\begin{align*}
    2x-4y-2z=6
\end{align*}
\begin{align*}
    x+2y-z=5m
\end{align*}
is inconsistent if:
\begin{multicols}{2}
\begin{enumerate}
    \item $k=3,m=\frac{4}{5}$
    \item $k\neq3,m\in R$
    \item $k\neq3,m\neq \frac{4}{5}$
    \item $k=3,m\neq \frac{4}{5}$
\end{enumerate}
\end{multicols}
\bigskip
\end{enumerate}
\end{document}
