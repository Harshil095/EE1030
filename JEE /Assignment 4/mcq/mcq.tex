%iffalse
\let\negmedspace\undefined
\let\negthickspace\undefined
\documentclass[journal,12pt,twocolumn]{IEEEtran}
\usepackage{cite}
\usepackage{amsmath,amssymb,amsfonts,amsthm}
\usepackage{algorithmic}
\usepackage{graphicx}
\usepackage{textcomp}
\usepackage{xcolor}
\usepackage{txfonts}
\usepackage{listings}
\usepackage{enumitem}
\usepackage{mathtools}
\usepackage{gensymb}
\usepackage{comment}
\usepackage[breaklinks=true]{hyperref}
\usepackage{tkz-euclide}
\usepackage{listings}
\usepackage{gvv}                                        
%\def\inputGnumericTable{}                                
\usepackage[latin1]{inputenc}                                
\usepackage{color}                                            
\usepackage{array}                                            
\usepackage{longtable}                                      
\usepackage{calc}                                            
\usepackage{multirow}                                        
\usepackage{hhline}                                          
\usepackage{ifthen}                                          
\usepackage{lscape}
\usepackage{tabularx}
\usepackage{array}
\usepackage{float}
\usepackage{multicol}


\newtheorem{theorem}{Theorem}[section]
\newtheorem{problem}{Problem}
\newtheorem{proposition}{Proposition}[section]
\newtheorem{lemma}{Lemma}[section]
\newtheorem{corollary}[theorem]{Corollary}
\newtheorem{example}{Example}[section]
\newtheorem{definition}[problem]{Definition}
\newcommand{\BEQA}{\begin{eqnarray}}
\newcommand{\EEQA}{\end{eqnarray}}
\newcommand{\define}{\stackrel{\triangle}{=}}
\theoremstyle{remark}
\newtheorem{rem}{Remark}

% Marks the beginning of the document
\begin{document}
\bibliographystyle{IEEEtran}
\vspace{3cm}

\title{2022-June Session-06-29-2022-shift-1}
\author{ee24btech11064 - Harshil Rathan}
\maketitle
\newpage
\bigskip

\renewcommand{\thefigure}{\theenumi}
`\renewcommand{\thetable}{\theenumi}
\begin{enumerate}
\item The probability that a randomly chosen $2\times 2$ matrix with all the entries from the set of first $10$ primes, is singular, is equal to:
\begin{multicols}{2}
\begin{enumerate}
    \item $\frac{133}{10^4}$
    \item $\frac{18}{10^3}$
    \item $\frac{19}{10^3}$
    \item $\frac{271}{10^4}$ 
\end{enumerate}
\end{multicols}
\bigskip
\item Let the solution of the differential equation $x\frac{dy}{dx}-y=\sqrt{y^2+16x^2}$, $y\brak{1}=3$ be $y=y\brak{x}$. Then $y\brak{2}$ is equal to :
\begin{multicols}{2}
\begin{enumerate}
    \item $15$
    \item $11$
    \item $13$
    \item $17$
\end{enumerate}
\end{multicols}
\bigskip
\item If the mirror image of the point $\brak{2,4,7}$ in the plane $3x-y+4z=2$ is $\brak{a,b,c}$, then $2a+b+2c$ is equal to:
\begin{multicols}{2}
\begin{enumerate}
       \item $54$
       \item $50$
       \item $-6$
       \item $-42$
\end{enumerate}
\end{multicols}
\bigskip
\item Let f: $R\rightarrow R$ be a function defined by:
\begin{align*}
f(x) =
\begin{cases}
\max \{ t^3 - 3t \} \quad ; \, t \leq x, \, x \leq 2 \\
x^2 + 2x - 6 \quad ; \, 2 < x < 3 \\
|x - 3| + 9 \quad ; \, 3 \leq x \leq 5 \\
2x + 1 \quad ; \, x > 5
\end{cases}
\end{align*} 
Where $[t]$ is the greatest integer less than or equal to t. Let m be the number of points where f is not differentiable and $I = \int_{-2}^{2} f(x) \, dx$. Then the ordered pair $\brak{m,I}$ is equal to:
\begin{multicols}{2}
\begin{enumerate}
    \item $\brak{3,\frac{27}{4}}$
    \item $\brak{3,\frac{23}{4}}$
    \item $\brak{4,\frac{27}{4}}$
    \item $\brak{4,\frac{23}{4}}$
\end{enumerate}
\end{multicols}
\bigskip
\item Let $\vec{a}=\alpha \hat{i}+3\hat{j}-\hat{k}$, $\vec{b}=3\hat{i}-\beta\hat{j}+4\hat{k}$ and $\vec{c}=\hat{i}+2\hat{j}-2\hat{k}$ where $\alpha,\beta \in R$, be three vectors. If the projection of $\vec{a}$ on $\vec{c}$ is $\frac{10}{3}$ and $\vec{b}\times \vec{c}=-6\hat{i}+10\hat{j}+7\hat{k}$, then the value of $\alpha+\beta$ is equal to:
\begin{multicols}{2}
\begin{enumerate}
    \item $3$
    \item $4$
    \item $5$
    \item $6$
\end{enumerate}
\end{multicols}
\bigskip
\item The area enclosed by $y^2=8x$ and $y=\sqrt{2}x$ that lies outside the triangle formed by $y=\sqrt{2}x,x=1,y=2\sqrt{2}$, is equal to :
\begin{multicols}{2}
\begin{enumerate}
    \item $\frac{16\sqrt{2}}{6}$
    \item $\frac{11\sqrt{2}}{6}$
    \item $\frac{13\sqrt{2}}{6}$
    \item $\frac{5\sqrt{2}}{6}$ 
\end{enumerate}
\end{multicols}
\bigskip
\item If the system of linear equations 
\begin{align*}
    2x+y-z=7
\end{align*}
\begin{align*}
    x-3y+2z=1
\end{align*}
\begin{align*}
    x+4y+\delta z=k
\end{align*}
where $\delta,k \in R$, has infinitely many solutions, then $\delta+k$ is equal to:
\begin{multicols}{2}
\begin{enumerate}
    \item $-3$
    \item $3$
    \item $6$
    \item $9$
\end{enumerate}
\end{multicols}
\bigskip
\item Let $\alpha$ and $\beta$ be the roots of the equation $x^2+\brak{2i-1}=0$. Then, the value of $|\alpha^8+\beta^8|$ is equal to:
\begin{multicols}{2}
\begin{enumerate}
       \item $50$
       \item $250$
       \item $1250$
       \item $1500$
\end{enumerate}
\end{multicols}
\bigskip
\item Let $\triangle \in \{\wedge, \vee, \Rightarrow, \Leftrightarrow\}$ be such that $\brak{p \wedge q}\triangle\brak{\brak{p\vee q}\Rightarrow q}$ is a tautology. Then $\triangle$ is equal to:
\begin{multicols}{2}
\begin{enumerate}
    \item $\wedge$
    \item $\vee$
    \item $\Rightarrow$
    \item $\Leftrightarrow$
\end{enumerate}
\end{multicols}
\bigskip
\item Let $A=[a_{ij}]$ be a square matrix of order 3 such that $a_{ij}=2^{j-i}$, for all $i,j=1,2,3$. Then, the matrix $A^2+A^3+\cdots+A^{10}$ is equal to:
\begin{multicols}{2}
\begin{enumerate}
    \item $\brak{\frac{3^{10}-3}{2}}$
    \item $\brak{\frac{3^{10}-1}{2}}$
    \item $\brak{\frac{3^{10}+1}{2}}$
    \item $\brak{\frac{3^{10}+3}{2}}$
\end{enumerate}
\end{multicols}
\bigskip
\item Let a set $A=A_1 \cup A_2 \cup \cdots \cup A_k$, where $A_i \cap A_j=\phi$ for $i\neq j,1\leq i,j\leq k$. Define the relation $R$ from $A$ to $A$ by $R=\{\brak{x,y}:y\in A_i$ if and only if $x \in A_i$, $1\leq i \leq k$\}. Then, $R$ is:
\begin{multicols}{2}
\begin{enumerate}
    \item reflexive, symmetric but not transitive
    \item reflexive, transitive but not symmetric
    \item reflexive but not symmetric and transitive
    \item an equivalence relation 
\end{enumerate}
\end{multicols}
\bigskip
\item Let $\{a_n\}_{n=0}^{\infty}$ be a sequence that $a_0=a_1=0$ and $a_{n+2}=2a_{n+1}-a_n+1$ for all $n \geq 0$.Then, $\sum_{n=2}^{\infty}\frac{a_n}{7^n}$ is equal to: 
\begin{multicols}{2}
\begin{enumerate}
    \item $\frac{6}{343}$
    \item $\frac{7}{216}$
    \item $\frac{8}{343}$
    \item $\frac{49}{216}$
\end{enumerate}
\end{multicols}
\bigskip
\item The distance between the two points $A$ and $A'$ which lie on $y=2$ such that both the line segments $AB$ and $A'B\brak{\text{where B is the point}\brak{2,3}}$subtend angle $\frac{\pi}{4}$ at the origin, is equal to :
\begin{multicols}{2}
\begin{enumerate}
    \item $10$
    \item $\frac{48}{5}$
    \item $\frac{52}{5}$
    \item $3$
\end{enumerate}
\end{multicols}
\bigskip
\item A wire of length $22m$ is to be cut into two pieces. One of the pieces is to be made into a square and the other into an equilateral triangle. Then, the length of the side of the equilateral triangle, so that the combined area of the square and the equilateral triangle is minimum, is:
\begin{multicols}{2}
\begin{enumerate}
    \item $\frac{22}{9+4\sqrt{3}}$
    \item $\frac{66}{9+\sqrt{3}}$
    \item $\frac{22}{4+9\sqrt{3}}$
    \item $\frac{66}{4+9\sqrt{3}}$
\end{enumerate}
\end{multicols}
\bigskip
\item The domain of the function $\cos^{-1}\brak{\frac{2\sin^{-1}\brak{\frac{1}{4x^2-1}}}{\pi}}$ is :
\begin{enumerate}
    \item $R-\{-\frac{1}{2},\frac{1}{2}\}$
    \item $(-\infty,-1]\cup [1,\infty)\cup \{0\}$
    \item $\brak{-\infty,\frac{-1}{2}}\cup\brak{\frac{1}{2},\infty}\cup \{0\}$
    \item $(-\infty,\frac{-1}{\sqrt{2}}]\cup [\frac{1}{\sqrt{2}},\infty)\cup \{0\}$
\end{enumerate}
\bigskip
\end{enumerate}
\end{document}
