%iffalse
\let\negmedspace\undefined
\let\negthickspace\undefined
\documentclass[journal,12pt,twocolumn]{IEEEtran}
\usepackage{cite}
\usepackage{amsmath,amssymb,amsfonts,amsthm}
\usepackage{algorithmic}
\usepackage{graphicx}
\usepackage{textcomp}
\usepackage{xcolor}
\usepackage{txfonts}
\usepackage{listings}
\usepackage{enumitem}
\usepackage{mathtools}
\usepackage{gensymb}
\usepackage{comment}
\usepackage[breaklinks=true]{hyperref}
\usepackage{tkz-euclide}
\usepackage{listings}
\usepackage{gvv}                                        
%\def\inputGnumericTable{}                                
\usepackage[latin1]{inputenc}                                
\usepackage{color}                                            
\usepackage{array}                                            
\usepackage{longtable}                                      
\usepackage{calc}                                            
\usepackage{multirow}                                        
\usepackage{hhline}                                          
\usepackage{ifthen}                                          
\usepackage{lscape}
\usepackage{tabularx}
\usepackage{array}
\usepackage{float}


\newtheorem{theorem}{Theorem}[section]
\newtheorem{problem}{Problem}
\newtheorem{proposition}{Proposition}[section]
\newtheorem{lemma}{Lemma}[section]
\newtheorem{corollary}[theorem]{Corollary}
\newtheorem{example}{Example}[section]
\newtheorem{definition}[problem]{Definition}
\newcommand{\BEQA}{\begin{eqnarray}}
\newcommand{\EEQA}{\end{eqnarray}}
\newcommand{\define}{\stackrel{\triangle}{=}}
\theoremstyle{remark}
\newtheorem{rem}{Remark}

% Marks the beginning of the document
\begin{document}
\bibliographystyle{IEEEtran}
\vspace{3cm}

\title{Assignment-1}
\author{ee24btech11064 - Harshil Rathan}
\maketitle
\newpage
\bigskip

\renewcommand{\thefigure}{\theenumi}
\renewcommand{\thetable}{\theenumi}
\section{\textcolor{red}{JEE Main / AIEEE}}
\begin{enumerate}
\item The maximum distance from origin of a point on the curve\\ $x = a \sin(t-b) \sin(\frac{at}{b})$\\ $y = a \cos(t-b) \cos(\frac{at}{b})$, both a, b $>$ 0 \hfill{[2002]}
\begin{enumerate}
    \item a-b
        \item a+b
	    \item $\sqrt{a^2+b^2}$
	        \item $\sqrt{a^2-b^2}$  \\
		\end{enumerate}
		\item If $2a+3b+6c=0$, ($a,b,c \in R$) then the quadratic equation $ax^2+bx+c$ has \hfill{[2002]}
		\begin{enumerate}
		    \item at least one root in $[0,1]$
		        \item at  least one root in $[2,3]$
			    \item at least one root in $[4,5]$
			        \item none of these \\
				\end{enumerate}
				\item If the function $f(x)=2x^3-9ax^2+12a^2x+1$, where a $>$0, attains its maximum and minimum at p and q respectively such that $p
				    ^2=q$, then a equals  \hfill{[2003]}
				    \begin{enumerate}
				           \item $\frac{1}{2}$
					          \item 3
						         \item 1
							        \item 2
								\end{enumerate}
								\item A point on the parabola $y^2=18x$ at which the ordinate increases at twice the rate of the abscissa is \hfill{[2004]}
								\begin{enumerate}
								    \item $(\frac{9}{8},\frac{9}{2})$
								        \item $(2,-4)$
									    \item $(\frac{-9}{8},\frac{9}{2})$
									        \item $(2,4)$
										\end{enumerate}
										\item A function $y=f(x)$ has a second order derivative $f"(x)=6(x-1)$.If its graph passes through the point (2,1) and at that point the tangent to the graph $y=3x-5$, then the function is \hfill{[2004]}
										\begin{enumerate}
										    \item $(x+1)^2$
										        \item $(x-1)^3$
											    \item $(x+1)^3$
											        \item $(x-1)^2$
												\end{enumerate}
												\item The normal to the curve $x=a(1+cos\theta)$$, $y=a sin$\theta$ at $\theta$ always passes through the fixed point\\ \hfill{[2004]}
												\begin{enumerate}
												    \item $\brak{a,a}$
												        \item $\brak{0,a}$
													    \item $\brak{0,0}$
													        \item $\brak{a,0}$
														\end{enumerate}
														\item If $2a+3b+6c=0$, then atleast one root of the equation $ax^2+bx+c$lies in the interval \hfill{[2004]}
														\begin{enumerate}
														    \item $\brak{1,3}$
														        \item $\brak{1,2}$
															    \item $\brak{2,3}$
															        \item $\brak{0,1}$
																\end{enumerate}  
																 \item Area of the greatest rectangle that can be inscribed in the ellipse $\frac{x^2}{a^2}+\frac{y^2}{b^2}=1$is
																 \begin{enumerate}
																     \item $2ab$
																         \item $ab$
																	     \item $\sqrt{ab}$
																	         \item $\frac{a}{b}$
																		 \end{enumerate}
																		 \item The normal to the curve $x=a \cos \theta$+ $\sin\theta$, y=$a\sin\theta - \cos \theta$ at any point $\theta$is such that
																		 \begin{enumerate}
																		     \item it passes through the origin
																		         \item it makes an angle $\frac{\pi}{2}+\theta$ with the x axis
																			     \item it passes through $(a\frac{\pi}{2},-a)$
																			         \item it is at a constant distance from the origin
																				 \end{enumerate}
																				 \end{enumerate}
																				 \section{\textcolor{red}{Integer value correct type}}
																				 \begin{enumerate}
																				 \item Let f be a function defined on $R$ (the set of all real numbers) such that $f'\brak{x}$=2010 $\brak{x-2009}$ $\brak{x-2010}^2$ $\brak{x-2011^3}$ $\brak{x-2012^4}$ for all x $\in R$ \\ If g is a function defined on $R$ with values in the interval $\brak{0,\infty}$ such that \\
																				   $f(x)$=$\ln{g(x)}$, for all x $\in R$\\ then the number of points in R at which $g$ has a local maximum is \hfill (2010)\\
																				   \item let $f$:$IR \rightarrow IR$ be defined as $f(x)$= $|x|$ + $|x^2-1|$. The total number of points at which $f$attains either a local maximum or a local minimum is\\ \hfill (2012) \\
																				   \item Let $p(x)$ be a real polynomial of least degree which has a local maximum at $x=1$ and local minimum at $x=3$.If $p(1)$=6 and $p(3)$=2, then $p'(0)$ is \hfill (2012) \\
																				   \item A vertical line passing through the point $\brak{h,0}$ intersects the ellipse $\frac{x^2}{4}$ + $\frac{y^2}{3}$ = 1 at the points P and Q .Let the tangents to the ellipse at P and Q meet at the point R. If $\Delta$(h)=area of the triangle PQR, $\Delta_1  = \max _{1/2\leq h \leq 1}\Delta\brak{h}$ and $\Delta_2 = min_{1/2\leq h \leq 1}\Delta\brak{h}$,then $\frac{8}{\sqrt{5}}\Delta_1 - 8\Delta_2$=\\\hfill $(JEEAdv.2013)$\\
																				   \item The slope of the tangent to the curve ($y-x^5)^2$ = $x(1+x^2)^2$ at the point $\brak{1,3}$ is \hfill ($JEE Adv. 2014)$ \\
																				   \item A cylindrical container is to be made from a certain solid material with the following constraints:It has a fixed inner volume of V $mm^3$, has a 2mm thick solid wall and is open at the top.The bottom of the container is a solid circular disc of thickness 2mm and is of radius equal to the outer radius of the container.\\If the volume of the material used to make the container is minimum when the inner radius of the container is 10mm, then the value of $\frac{V}{250\pi}$ is \hfill ($JEE Adv. 2015)$ \\  
																				   \end{enumerate}
																				   \end{document}
