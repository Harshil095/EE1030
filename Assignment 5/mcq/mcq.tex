%iffalse
\let\negmedspace\undefined
\let\negthickspace\undefined
\documentclass[journal,12pt,twocolumn]{IEEEtran}
\usepackage{cite}
\usepackage{amsmath,amssymb,amsfonts,amsthm}
\usepackage{algorithmic}
\usepackage{graphicx}
\usepackage{textcomp}
\usepackage{xcolor}
\usepackage{txfonts}
\usepackage{listings}
\usepackage{enumitem}
\usepackage{mathtools}
\usepackage{gensymb}
\usepackage{comment}
\usepackage[breaklinks=true]{hyperref}
\usepackage{tkz-euclide}
\usepackage{listings}
\usepackage{gvv}                                        
%\def\inputGnumericTable{}                                
\usepackage[latin1]{inputenc}                                
\usepackage{color}                                            
\usepackage{array}                                            
\usepackage{longtable}                                      
\usepackage{calc}                                            
\usepackage{multirow}                                        
\usepackage{hhline}                                          
\usepackage{ifthen}                                          
\usepackage{lscape}
\usepackage{tabularx}
\usepackage{array}
\usepackage{float}
\usepackage{multicol}


\newtheorem{theorem}{Theorem}[section]
\newtheorem{problem}{Problem}
\newtheorem{proposition}{Proposition}[section]
\newtheorem{lemma}{Lemma}[section]
\newtheorem{corollary}[theorem]{Corollary}
\newtheorem{example}{Example}[section]
\newtheorem{definition}[problem]{Definition}
\newcommand{\BEQA}{\begin{eqnarray}}
\newcommand{\EEQA}{\end{eqnarray}}
\newcommand{\define}{\stackrel{\triangle}{=}}
\theoremstyle{remark}
\newtheorem{rem}{Remark}

% Marks the beginning of the document
\begin{document}
\bibliographystyle{IEEEtran}
\vspace{3cm}

\title{2023-April Session-04-12-2023-shift-1}
\author{ee24btech11064 - Harshil Rathan}
\maketitle
\newpage
\bigskip

\renewcommand{\thefigure}{\theenumi}
`\renewcommand{\thetable}{\theenumi}
\begin{enumerate}
\item The number of five digit numbers, greater than $40000$ and divisible by $5$, which can be formed using the digits $0, 1, 3, 5, 7$ and $9$ without repetition, is equal to 
\begin{multicols}{2}
\begin{enumerate}
    \item $120$
    \item $132$
    \item $72$
    \item $96$ 
\end{enumerate}
\end{multicols}
\bigskip
\item Let $\alpha,\beta$ be the roots of the quadratic equation $x^2+\sqrt{6}x+3=0$. Then $\frac{\alpha^{23}+\beta^{23}+\alpha^{14}+\beta^{14}}{\alpha^{15}+\beta^{15}+\alpha^{10}+\beta^{10}}$ is equal to
\begin{multicols}{2}
\begin{enumerate}
    \item $729$
    \item $72$
    \item $81$
    \item $9$
\end{enumerate}
\end{multicols}
\bigskip
\item Let $<a_n>$ be a sequence such that 
\begin{align*}
    a_1+a_2+\cdots +a_n=\frac{n^2+3n}{\brak{n+1}\brak{n+2}}.
\end{align*}
If, $28\sum_{k=1}^{10} \frac{1}{a_k} = p_1 \cdot p_2 \cdot p_3 \cdots p_m$, where $p_1, p_2, \cdots, p_m$
are the first m prime numbers, then m is equal to:
\begin{multicols}{2}
\begin{enumerate}
       \item $7$
       \item $6$
       \item $5$
       \item $8$
\end{enumerate}
\end{multicols}
\bigskip
\item Let the lines $l_1: \frac{x+5}{3}=\frac{y+4}{1}=\frac{z-\alpha}{-2}$ and $l_2: 3x+2y+z-2=0=x-3y+2z-13$ be coplanar. If the point $P\brak{a,b,c}$ on $l_1$ is nearest to the point $Q\brak{-4,-3,2}$, then $|a|+|b|+|c|$ is equal to
\begin{multicols}{2}
\begin{enumerate}
       \item $12$
       \item $14$
       \item $10$
       \item $8$
\end{enumerate}
\end{multicols}
\bigskip
\item Let $P\brak{\frac{2\sqrt{3}}{7},\frac{6}{\sqrt{7}}}$, $Q$, $R$ and $S$ be four points on th ellipse $9x^2+4y^2=36$. Let $PQ$ and $RS$ be mutually perpendicular and pass through the origin. If $\frac{1}{\brak{PQ}^2}+\frac{1}{\brak{RS}^2}=\frac{p}{q}$, where $p$ and $q$ are co-prime, then $p+q$ is equal to
\begin{multicols}{2}
\begin{enumerate}
    \item $143$
    \item $137$
    \item $157$
    \item $147$
\end{enumerate}
\end{multicols}
\bigskip
\item Let $a, b, c$ be three distinct real numbers, none equal to one. If the vectors $a\hat{i}+\hat{j}+\hat{k},\hat{i}+b\hat{j}+\hat{k}$ and $\hat{i}+\hat{j}+c\hat{k}$ are coplanar, then $\frac{1}{1-a}+\frac{1}{1-b}+\frac{1}{1-c}$ is equal to
\begin{multicols}{2}
\begin{enumerate}
    \item $1$
    \item $-1$
    \item $-2$
    \item $2$
\end{enumerate}
\end{multicols}
\bigskip
\item If the local maximum value of the function $f\left( x \right) = \left( \frac{\sqrt{3e}}{2 \sin{x}} \right)^{\sin^2{x}}$, $x\in \brak{0,\frac{\pi}{2}}$, is $\frac{k}{e}$, then $\brak{\frac{k}{e}}^8+\frac{k^8}{e^5}+k^8$ is equal to 
\begin{multicols}{2}
\begin{enumerate}
    \item $e^5+e^6+e^{11}$
    \item $e^3+e^5+e^{11}$
    \item $e^3+e^6+e^{11}$
    \item $e^3+e^6+e^{10}$
\end{enumerate}
\end{multicols}
\bigskip
\item Let $D$ be the domain of the function $f\brak{x}=\sin^{-1}{\brak{\log_{3x}\brak{\frac{6+2\log_3 x}{-5x}}}}$. If the range of the function $g:D\rightarrow R$ defined by $g\brak{x}=x-[x],\brak{[x]\text{ is the greatest  integer function}}$, is $\brak{\alpha,\beta}$, then $\alpha+\frac{5}{\beta}$ is equal to 
\begin{multicols}{2}
\begin{enumerate}
       \item $46$
       \item $135$
       \item $136$
       \item $45$
\end{enumerate}
\end{multicols}
\bigskip
\item Let $y=y\brak{x},y>0$, be a solution curve of the differential equation $\brak{1+x^2}dy=y\brak{x-y}dx$. If $y\brak{0}=1$ and $y\brak{2\sqrt{2}}=\beta$, then
\begin{multicols}{2}
\begin{enumerate}
    \item $e^{3\beta^{-1}}=e\brak{3+2\sqrt{2}}$
    \item $e^{\beta^{-1}}=e^{-2}\brak{5+\sqrt{2}}$
    \item $e^{\beta^{-1}}=e^{-2}\brak{3+2\sqrt{2}}$
    \item $e^{3\beta^{-1}}=e\brak{5+\sqrt{2}}$
\end{enumerate}
\end{multicols}
\bigskip
\item Among the two statements 
\begin{align*}
    \brak{S1}:\brak{p\Rightarrow q}\land \brak{q\land\brak{\neg q}} \text{is a contradiction}
\end{align*}
\begin{align*}
    \brak{S2}:\brak{p\land q}\lor \brak{\brak{\neg p}\land q}\lor \brak{p \land \brak{\neg q}\lor \brak{\brak{\neg p}}\land\brak{\neg q}}
\end{align*}
is a tautology 
\begin{multicols}{2}
\begin{enumerate}
    \item only$\brak{S2}$ is true
    \item only $\brak{S1}$ is true 
    \item both are false 
    \item both are true 
\end{enumerate}
\end{multicols}
\bigskip
\item Let $\lambda \in Z$, $\vec{a}=\lambda \hat{i}+\hat{j}-\hat{k}$ and $\vec{b}=3\hat{i}-\hat{j}+2\hat{k}$. Let $\vec{c}$ be a vector such that $\brak{\vec{a}+\vec{b}+\vec{c}}\times \vec{c=0},\vec{a}\cdot \vec{c}=-17$ and $\vec{b}\cdot \vec{c}=-20$. Then $|\vec{c}\times \brak{\lambda \hat{i}+\hat{j}+\hat{k}}|^2$ is equal to 
\begin{multicols}{2}
\begin{enumerate}
         \item $46$
       \item $135$
       \item $136$
       \item $45$
\end{enumerate}
\end{multicols}
\bigskip
\item The sum, of the coefficients of the first 50 terms in the binomial expansion of $\brak{1-x}^{100}$, is equal to 
\begin{multicols}{2}
\begin{enumerate}
    \item $-^{101}C_{50}$
    \item $^{99}C_{49}$
    \item $-^{99}C_{49}$
    \item $^{101}C_{50}$
\end{enumerate}
\end{multicols}
\bigskip
\item The area of the region enclosed by the curve $y=x^3$ and its tangent at the point $\brak{-1,-1}$ is
\begin{multicols}{2}
\begin{enumerate}
    \item $\frac{27}{4}$
    \item $\frac{19}{4}$
    \item $\frac{23}{4}$
    \item $\frac{31}{4}$
\end{enumerate}
\end{multicols}
\bigskip
\item Let $A=\myvec{1&\frac{1}{51}\\0 &1}$. If $B=\myvec{1&2\\-1&-1}A\myvec{-1&-2\\1&1}$, then the sum of all the elements of the matrix $\sum_{n=1}^{50}B^n$ is equal to 
\begin{multicols}{2}
\begin{enumerate}
       \item $100$
       \item $50$
       \item $75$
       \item $125$
\end{enumerate}
\end{multicols}
\bigskip
\item Let the plane $P:4x-y+z=10$ be rotated by an angle $\frac{\pi}{2}$ about its line of intersection with the plane $x+y-z=4$. If $\alpha$ is the distance of the point $\brak{2,3,-4}$ from the new position of the plane $P$, then $35\alpha$ is 
\begin{enumerate}
      \item $90$
       \item $85$
       \item $105$
       \item $126$
\end{enumerate}
\bigskip
\end{enumerate}
\end{document}
