%iffalse
\let\negmedspace\undefined
\let\negthickspace\undefined
\documentclass[journal,12pt,twocolumn]{IEEEtran}
\usepackage{cite}
\usepackage{amsmath,amssymb,amsfonts,amsthm}
\usepackage{algorithmic}
\usepackage{graphicx}
\usepackage{textcomp}
\usepackage{xcolor}
\usepackage{txfonts}
\usepackage{listings}
\usepackage{enumitem}
\usepackage{mathtools}
\usepackage{gensymb}
\usepackage{comment}
\usepackage[breaklinks=true]{hyperref}
\usepackage{tkz-euclide}
\usepackage{listings}
\usepackage{gvv}                                        
%\def\inputGnumericTable{}                                
\usepackage[latin1]{inputenc}                                
\usepackage{color}                                            
\usepackage{array}                                            
\usepackage{longtable}                                      
\usepackage{calc}                                            
\usepackage{multirow}                                        
\usepackage{hhline}                                          
\usepackage{ifthen}                                          
\usepackage{lscape}
\usepackage{tabularx}
\usepackage{array}
\usepackage{float}
\usepackage{multicol}


\newtheorem{theorem}{Theorem}[section]
\newtheorem{problem}{Problem}
\newtheorem{proposition}{Proposition}[section]
\newtheorem{lemma}{Lemma}[section]
\newtheorem{corollary}[theorem]{Corollary}
\newtheorem{example}{Example}[section]
\newtheorem{definition}[problem]{Definition}
\newcommand{\BEQA}{\begin{eqnarray}}
\newcommand{\EEQA}{\end{eqnarray}}
\newcommand{\define}{\stackrel{\triangle}{=}}
\theoremstyle{remark}
\newtheorem{rem}{Remark}

% Marks the beginning of the document
\begin{document}
\bibliographystyle{IEEEtran}
\vspace{3cm}

\title{2020-January Session-07-01-2020-shift-1}
\author{ee24btech11064 - Harshil Rathan}
\maketitle
\newpage
\bigskip

\renewcommand{\thefigure}{\theenumi}
`\renewcommand{\thetable}{\theenumi}
\begin{enumerate}
\item $lim_{x \rightarrow 2}$ $\frac{3^x+\frac{27}{3^x}-12}{\frac{1}{3^\frac{x}{2}}-\frac{3}{3^x}}$ is equal to.\\
\item If variance of first $n$ natural numbers is $10$ and variance of first $m$ even natural numbers is $16$, $m + n$ is equal to\\
\item  If the sum of the coefficients of all even powers of $x$ in the product
\begin{align*}
    \brak{1+x+x^2+x^3... . +x^{2n}}\brak{1-x+x^2-x^3... . +x^{2n}}
\end{align*}
is $61$, then $n$ is equal to \\
\item Let $S$ be the set of points where the function, $f\brak{x}=|2-|x-3||$, $x \in R$, is not differentiable. Then, the value of $\sum x \in S, f\brak{f\brak{x}}$ is equal to
\begin{figure}[h!]
   \centering
   \includegraphics[width=\linewidth]{q24.png}
   \caption{}
\end{figure}
\item Let $A\brak{1,0}$, $B\brak{6,2}$, $C\brak{\frac{3}{2},6}$ be the vertices of a triangle $ABC$. If $P$ is a point inside the triangle $ABC$ such that the triangle $APC,APB$ and $BPC$ have equal areas, then the length of the line segment $PQ$, where $Q$ is the point $\brak{\frac{-7}{6},\frac{-1}{3}}$, is 
\end{enumerate}
\end{document}
