%iffalse
\let\negmedspace\undefined
\let\negthickspace\undefined
\documentclass[journal,12pt,twocolumn]{IEEEtran}
\usepackage{cite}
\usepackage{amsmath,amssymb,amsfonts,amsthm}
\usepackage{algorithmic}
\usepackage{graphicx}
\usepackage{textcomp}
\usepackage{xcolor}
\usepackage{txfonts}
\usepackage{listings}
\usepackage{enumitem}
\usepackage{mathtools}
\usepackage{gensymb}
\usepackage{comment}
\usepackage[breaklinks=true]{hyperref}
\usepackage{tkz-euclide}
\usepackage{listings}
\usepackage{gvv}                                        
%\def\inputGnumericTable{}                                
\usepackage[latin1]{inputenc}                                
\usepackage{color}                                            
\usepackage{array}                                            
\usepackage{longtable}                                      
\usepackage{calc}                                            
\usepackage{multirow}                                        
\usepackage{hhline}                                          
\usepackage{ifthen}                                          
\usepackage{lscape}
\usepackage{tabularx}
\usepackage{array}
\usepackage{float}
\usepackage{multicol}


\newtheorem{theorem}{Theorem}[section]
\newtheorem{problem}{Problem}
\newtheorem{proposition}{Proposition}[section]
\newtheorem{lemma}{Lemma}[section]
\newtheorem{corollary}[theorem]{Corollary}
\newtheorem{example}{Example}[section]
\newtheorem{definition}[problem]{Definition}
\newcommand{\BEQA}{\begin{eqnarray}}
\newcommand{\EEQA}{\end{eqnarray}}
\newcommand{\define}{\stackrel{\triangle}{=}}
\theoremstyle{remark}
\newtheorem{rem}{Remark}

% Marks the beginning of the document
\begin{document}
\bibliographystyle{IEEEtran}
\vspace{3cm}

\title{2020-January Session-07-01-2020-shift-1}
\author{ee24btech11064 - Harshil Rathan}
\maketitle
\newpage
\bigskip

\renewcommand{\thefigure}{\theenumi}
`\renewcommand{\thetable}{\theenumi}
\begin{enumerate}
\item Let $x^k + y^k = a^k$, $\brak{a,k>0}$ and $\brak{\frac{dy}{dx}}$+$\brak{\frac{y}{x}}^\frac{1}{3}$ = $0$, then k is 
\begin{multicols}{2}
\begin{enumerate}
    \item $\frac{1}{3}$
    \item $\frac{3}{2}$
    \item $\frac{2}{3}$
    \item $\frac{4}{3}$ 
\end{enumerate}
\end{multicols}
\item Let the function, $f:[-7,0]\rightarrow R$ be continuous on $[-7,0]$ and differentiable on $\brak{-7,0}$. If $f\brak{-7}=-3$ and $f'\brak{x} \leq 2$, for all $x \in \brak{-7,0}$, then for all such functions f, $f\brak{-1}+f\brak{0}$ lies in the interval:
\begin{multicols}{2}
\begin{enumerate}
    \item $[-6,20]$
    \item $(-\infty,20]$
    \item $(-\infty,11]$
    \item $[-3,11]$
\end{enumerate}
\end{multicols}
\item If $y = y\brak{x}$ is the solution of the differential equation $e^y\brak{\frac{dy}{dx}}-1=e^x$ such that $y\brak{0}=0$, then $y\brak{1}$ is equal to
\begin{multicols}{2}
\begin{enumerate}
       \item $\log_e2$
       \item $2e$
       \item $2+\log_e2$
       \item $1+\log_e2$
\end{enumerate}
\end{multicols}
\item Five numbers are in A.P, whose sum is $25$ and product is $2520$. If one of these five numbers is $\frac{-1}{2}$, then the greatest number amongst them is
\begin{multicols}{2}
\begin{enumerate}
    \item $16$
    \item $27$
    \item $7$
    \item $\frac{21}{2}$
\end{enumerate}
\end{multicols}
\item If the system of linear equations
\begin{align*}
    2x+2ay+az=0
\end{align*}
\begin{align*}
    2x+3by+bz=0
\end{align*}
\begin{align*}
    2x+4cy+cz=0,
\end{align*}
where $a,b,c \in R $ are non-zero and distinct, has non-zero solution, then
\begin{multicols}{2}
\begin{enumerate}
    \item $a+b+c=0$
    \item $a,b,c$ are in A.P
    \item $\frac{1}{a},\frac{1}{b},\frac{1}{c}$ are in A.P
    \item $a,b,c$ are in G.P
\end{enumerate}
\end{multicols}
\end{enumerate}
\end{document}
